\documentclass{fo-article}
\usepackage{fo-symbols}

\begin{document}

\author{ds Lee}
\title{文档学格式规范}
\maketitle

\chapter{基本格式}
\section{概述}

本文档将阐述如何撰写符合规范的文档。

文档学提供 \LaTeX\ 模板。当前使用的模板是 fo-article.

\section{中文文档的字体一致性}

文档学接受简体中文和繁体中文文档。但不得在同一文档中混用两种字形。

文档学对于使用文字字形的政治色彩不予考虑。

\section{中英混排的字体一致性}

文档学接受中英混排的文档,但是需要遵循以下规范:

\li
\item 分离全角与半角字符。任何全角与半角字符之间都要用空格分开。
\lix

\chapter{符号}
\section{标点符号}

\subsection{定义}

中文标点符号遵循如下定义:

\descc{align=left,labelwidth=1cm}
\item [。] 句号。表达句子的结束。句子的结束位于主语功能的结束。
\item [,] 逗号。表达句子中的停顿。
\item [?] 问号。表达疑问、设问或反问。
\item [!] 叹号。加强句子的语气。
\item [:] 冒号。表达解释说明、列表或引导人物所说的话。
\item [;] 分号。表达句子之间的并列关系。
\item [「」] 引号。表达引用或讽刺地强调。
\item [·] 分隔号。用于人名之间的分隔。
\item [《》] 书名号。用于表达文献名。
\item [——] 破折号。用于表达语气的转折或语法结构的转折,或解释说明。
\item [()] 括号。用于补充说明,或表示对引用句子的结构补充。
\item [……] 省略号。用于表达语气的断续或文字的省略。
\descx

其他的符号,如连接号、方括号等,由于使用不便而不在文档学中使用。

\subsection{引号规范}

推荐使用直角引号。双引号和单引号也可使用。但是不得在同一文档中混用两种引号。

\subsection{话语标点}

在表述他人的话语时,需要遵循以下标点规范:

\li 
\item 主语在前。此时应在主语后用冒号引导话语,并用引号包围。如:
甲说:「这是一句话。」
\item 主语在中。此时,前半部分用引号包围,再插入主语,并用逗号引导后半部分。如:
「很有意思。」乙说,「我从未见过这样的景象。」
\item 主语在后。此时,仅用引号包围话语即可。如:
「这是一句话。」丙说道。
\item 大段引用时,不必在段后闭合引号。只需要在每段段前使用引号引导,并在引用结束后闭合。
\lix

\subsection{标点滥用}

\li
\item 逗号滥用。不应大段使用逗号。除非主语功能延续到整段,必须通过句号结束句子。
\item 句号赘余。要表达语气断续或省略,应使用省略号而非连续使用三个或多个句号。
\item 逗号赘余。见句号赘余。
\item 顿号赘余。见句号赘余。
\item 破折号滥用。对于不强烈的转折,推荐使用逗号或句号。对于解释说明,如果没有语法结构或语气的转折,推荐使用冒号。
\item 省略号滥用。省略号最多连续使用 12 个点,表示整段或整篇文章的省略。
\item 引号多层嵌套。引号不应超过两层,在第二层之内不应使用需要引号的表达。
\lix

\section{特殊符号}

\subsection{指示性符号}

这些符号使用了 Unicode 6.0 标准。
以下列出了具有特殊意义的符号。

\desc
\item [\warning] 用于指示一个需要注意的内容。
\item [\infpaper] 用于指示该文档是客观事实。
\item [\medics] 用于指示该文档是医学文档。
\item [\technology] 用于指示该文档是科技文档。
\descx

\subsection{表情符号}

\stress{严重不推荐}在文档内使用表情符号。除非该文档是关于表情符号的。

\end{document}